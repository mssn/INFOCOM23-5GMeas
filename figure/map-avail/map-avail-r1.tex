% Template for a CDF graph
%
% Author: Zengwen Yuan
% Version: 1.0  2017-03-08 init version
% Note: to use package fontspec,
% use XeLaTeX to compile

\documentclass{article}
\usepackage{siunitx}
% \usepackage{tikz}
\usepackage{pgfplots}
\usepackage{pgfplotstable}
%\usepackage{verbatim}
\usepackage{sansmath}
\usepackage[active,tightpage]{preview}
\PreviewEnvironment{tikzpicture}
\setlength\PreviewBorder{1pt}
\usetikzlibrary{patterns}

\newlength\figureheight
\newlength\figurewidth
\setlength\figureheight{11cm}
\setlength\figurewidth{10cm}

\pgfplotsset{compat=newest}

%\usepackage{fontspec}
%\setmainfont[
%BoldFont={Arial Bold},
%ItalicFont={Arial Italic},
%BoldItalicFont={Arial Bold Italic}
%]{Arial}

% # user-study-sync-overhead-context-tx-v2.txt
%\pgfplotstableread{
%0   0
%1000   0.1
%5000   0.8
%10000   0.95
%20000   1
%}{\fakedata}

\makeatletter
\pgfplotsset{
	my filter/.style args={every#1between#2and#3}{%
		/pgfplots/x filter/.append code={%
			\ifnum\coordindex<#2%
			% Nothing
			\else% Did we pass #3?
			\ifnum\coordindex>#3%
			%Nothing
			\else% Ok filter is on, don't disturb \pgfmathresult for convenience
			\pgfmathsetmacro\temp{int(mod(\coordindex,#1))}%
			\ifnum0=\temp\relax% Are we on the nth point?
			% Yes do nothing let it pass
			\else% discard it
			\let\pgfmathresult\pgfutil@empty
			\fi%
			\fi%
			\fi%
}}}
\makeatother

\pgfplotsset{
	compat=newest,
	legend image code/.code={
		\draw[mark repeat=2,mark phase=2]
		plot coordinates {
			(0cm,0cm)
			(0.3cm,0cm)        %% default is (0.3cm,0cm)
			(0.6cm,0cm)         %% default is (0.6cm,0cm)
		};%
	}
}

% \definecolor{palette1}{RGB}{215,25,28}
% \definecolor{palette2}{RGB}{253,174,97}
% % \definecolor{palette3}{RGB}{255,255,191}
% \definecolor{palette3}{RGB}{208,28,139}
% \definecolor{palette4}{RGB}{184,225,134}
% % \definecolor{palette4}{RGB}{171,221,164}
% \definecolor{palette5}{RGB}{43,131,186}
% \definecolor{palette6}{RGB}{208,28,139}

% green -- blueish
% \definecolor{palette1}{RGB}{0,109,44}
% \definecolor{palette2}{RGB}{44,162,95}
% \definecolor{palette3}{RGB}{67,162,202}
% \definecolor{palette4}{RGB}{8,104,172}
% \definecolor{palette5}{RGB}{8,81,156}

\definecolor{myred}{RGB}{202,0,32}
\definecolor{myorange}{RGB}{244,165,130}
\definecolor{myviolet}{RGB}{194,165,207}
\definecolor{mycyan}{RGB}{146,197,222}
\definecolor{myblue}{RGB}{5,113,176}
\definecolor{mygreen}{RGB}{127,191,123}
\definecolor{mytile}{RGB}{27,120,55}
\definecolor{myblack}{RGB}{60,60,60}
\definecolor{mywhite}{RGB}{250,250,250}
\definecolor{mygray2}{RGB}{200,200,200}
\definecolor{mygray}{RGB}{220,220,220}
\definecolor{subregioncolor2}{RGB}{27,120,55}
\definecolor{subregioncolor1}{RGB}{0,255,0}
% \definecolor{palette3}{RGB}{247,247,247}

% 1.97km x 2.22km 
%% X:Y = 23:20, but width:height = 4cm:4.50cm 
%% So  marker's width:height = 1: (4.53/4 * 23/20) = 1: 1.3023
%\pgfdeclareplotmark{mymark}
%{%
%	\path[fill=white,postaction={pattern = north east lines, pattern color=myblue}] (-\pgfplotmarksize,-\pgfplotmarksize) rectangle (\pgfplotmarksize,1.3023*\pgfplotmarksize);
%}
%\pgfdeclareplotmark{mygridmark}
%{%
%	\path[] (-\pgfplotmarksize,-\pgfplotmarksize) rectangle (\pgfplotmarksize,1.3023*\pgfplotmarksize);
%}

\pgfdeclareplotmark{mygridmark}
{%
%	\path[fill=mapped color] (-\pgfplotmarksize,-\pgfplotmarksize) rectangle (\pgfplotmarksize,1.3023*\pgfplotmarksize);
\path[fill=mapped color, line width=0.05pt] (-\pgfplotmarksize,-\pgfplotmarksize) rectangle (\pgfplotmarksize,1.3023*\pgfplotmarksize);
}

\begin{document}
	
	\begin{tikzpicture}
		\begin{axis}[
		restrict x to domain=-86.168:-86.145,
		ymax    = 39.781,
		ymin    = 39.761,
		xmin = -86.168,
		xmax = -86.145,
		width   = 4cm,
		height  = 4.5cm,
		scale only axis = true,
		xtick align = inside,
		%axis line style={draw=mygray},
		axis line style={draw=black},
		ticks=none,
		colorbar,
%		colormap/hot,
		%colormap={whiteblue}{color=(white) color=(mygray) color=(120,120,250) color=(blue)},
		%colormap={whiteblue}{color=(white) color=(mygray2) color=(blue)},
           	% colormap={gb}{color=(240, 240, 240) color=(white) color=(green)},
		colormap={gray2black}{
			rgb255=(250, 250, 250)
			%rgb255=(135,206,250)
			%srgb255=(0,0,240)
			rgb255=(0,0,255)
			%(202,0,32)
			%(10,10,10)
		},
%	colormap={wbb}{
%		color(0)=(white)
%		color(900)=(blue)
%		color(1000)=(rgb:red,11;green,11;blue,69)
%	},
%	colormap/hot2,
		%colormap/gray2black,
		%colormap/whiteblue,
		colorbar horizontal,
		colorbar style={
			xtick={0, 0.25, 0.5, 0.75, 1},
			%xticklabels={0, 25\%, 50\%, 75\%, 100\%},
			xticklabels={},
			%ylabel={mw ratio},
			%				yticklabel style={text width=0.1cm,}
			at={(0.05,0.9)},
			%at={(0.05,1.1)},
			ymin=0,
			ymax=1,
			%				anchor=north,
			%height=0.15cm,
			height=0cm,
			%width=0.8*\pgfkeysvalueof{/pgfplots/parent axis width},
			width=0,
		}
		]
		
			\addplot[
			only marks,
			mark=mygridmark,
			mark size=1.28pt,
			scatter,
			scatter src=explicit,
			point meta=explicit,
			]
			%		file {data.dat};
			table[
			y = grid_lon,
			x = grid_lat,
			meta = 5g_ratio,
			col sep=comma,
			]
			{A-R1_connectivity_ratio_pixel5_20210405_20211226_310410_0.0005_0_0_0.95_0.05_grid_passed.csv};
			%{A-R1_connectivity_ratio_pixel5_0405-1009_310410_0.0005_0_0_0.95_0.05_grid_passed_5gm.csv};
			%{A-R1_connectivity_ratio_pixel5_0405-0930_310410_0.0005_0_0_0.95_0.05_grid_passed_5gm.csv};
			%{A-R1_connectivity_ratio_pixel5_0407_0807_310410_0.0005_0_0_0.95_0.05_grid_passed_5gm.csv};
			% {A-R1_connectivity_ratio_pixel5_0407_0807_310410_0.0005_0_0_area_type_matched_10_10_5gm.csv};
			
% R1: 39.761	39.781	-86.168	-86.145
% 1.97km x 2.22km
					
			%\pgfplotsextra{
				%				%subregion2 R1B
%				\draw[line width=0.3mm, subregioncolor2,-,>=latex] (axis cs:-86.166,39.7725) to (axis cs:-86.166,39.7665);
%				\draw[line width=0.3mm, subregioncolor2,-,>=latex] (axis cs:-86.166,39.7665) to (axis cs:-86.161,39.7665);
%				\draw[line width=0.3mm, subregioncolor2,-,>=latex] (axis cs:-86.161,39.7665) to (axis cs:-86.161,39.761);
%				\draw[line width=0.3mm, subregioncolor2,-,>=latex] (axis cs:-86.161,39.761) to (axis cs:-86.154,39.761);
%				\draw[line width=0.3mm, subregioncolor2,-,>=latex] (axis cs:-86.154,39.761) to (axis cs:-86.154,39.7665);
%				\draw[line width=0.3mm, subregioncolor2,-,>=latex] (axis cs:-86.154,39.7665) to (axis cs:-86.149,39.7665);
%				\draw[line width=0.3mm, subregioncolor2,-,>=latex] (axis cs:-86.149,39.7665) to (axis cs:-86.149,39.7705);
%				\draw[line width=0.3mm, subregioncolor2,-,>=latex] (axis cs:-86.149,39.7705) to (axis cs:-86.151,39.7705);
%				\draw[line width=0.3mm, subregioncolor2,-,>=latex] (axis cs:-86.151,39.7705) to (axis cs:-86.151,39.7725);
%				\draw[line width=0.3mm, subregioncolor2,-,>=latex] (axis cs:-86.151,39.7725) to (axis cs:-86.15,39.7725) ;
%				\draw[line width=0.3mm, subregioncolor2,-,>=latex] (axis cs:-86.15,39.7725) to (axis cs:-86.15,39.775);
%				\draw[line width=0.3mm, subregioncolor2,-,>=latex] (axis cs:-86.15,39.775) to (axis cs:-86.157,39.775);
%				\draw[line width=0.3mm, subregioncolor2,-,>=latex] (axis cs:-86.157,39.775) to (axis cs:-86.157,39.7725);
%				\draw[line width=0.3mm, subregioncolor2,-,>=latex] (axis cs:-86.157,39.7725) to (axis cs:-86.166,39.7725);
%				%\node[text=subregioncolor2] at (axis cs:-86.164,39.768) {R1B};
% R1A
\iffalse
\draw[line width=0.3mm, subregioncolor1,-,>=latex] (axis cs:-86.1668,39.7744) to (axis cs:-86.1668,39.768);
\draw[line width=0.3mm, subregioncolor1,-,>=latex] (axis cs:-86.1668,39.768) to (axis cs:-86.1675,39.768);
\draw[line width=0.3mm, subregioncolor1,-,>=latex] (axis cs:-86.1675,39.768) to (axis cs:-86.1675,39.7645);
\draw[line width=0.3mm, subregioncolor1,-,>=latex] (axis cs:-86.1675,39.7645) to (axis cs:-86.166,39.7645);
\draw[line width=0.3mm, subregioncolor1,-,>=latex] (axis cs:-86.166,39.7645) to (axis cs:-86.166,39.766);
\draw[line width=0.3mm, subregioncolor1,-,>=latex] (axis cs:-86.166,39.766) to (axis cs:-86.162,39.766);
\draw[line width=0.3mm, subregioncolor1,-,>=latex] (axis cs:-86.162,39.766) to (axis cs:-86.162,39.762);
\draw[line width=0.3mm, subregioncolor1,-,>=latex] (axis cs:-86.162,39.762) to (axis cs:-86.1678,39.762);
\draw[line width=0.3mm, subregioncolor1,-,>=latex] (axis cs:-86.1678,39.762) to (axis cs:-86.1678,39.761);
%\draw[line width=0.3mm, subregioncolor1,-,>=latex] (axis cs:-86.1678,39.761) to (axis cs:-86.1535,39.761);
\draw[line width=0.3mm, subregioncolor1,-,>=latex] (axis cs:-86.1678,39.761) to (axis cs:-86.1495,39.761);
\draw[line width=0.3mm, subregioncolor1,-,>=latex] (axis cs:-86.1495,39.761) to (axis cs:-86.1495,39.762);
\draw[line width=0.3mm, subregioncolor1,-,>=latex] (axis cs:-86.1535,39.762) to (axis cs:-86.1495,39.762);
%\draw[line width=0.3mm, subregioncolor1,-,>=latex] (axis cs:-86.1535,39.761) to (axis cs:-86.1535,39.762);
%\draw[line width=0.3mm, subregioncolor1,-,>=latex] (axis cs:-86.1678,39.761) to (axis cs:-86.1495,39.761);
%\draw[line width=0.3mm, subregioncolor1,-,>=latex] (axis cs:-86.1495,39.761) to (axis cs:-86.151,39.762);
%\draw[line width=0.3mm, subregioncolor1,-,>=latex] (axis cs:-86.151,39.762) to (axis cs:-86.1535,39.762);
\draw[line width=0.3mm, subregioncolor1,-,>=latex] (axis cs:-86.1535,39.762) to (axis cs:-86.1535,39.7645);
\draw[line width=0.3mm, subregioncolor1,-,>=latex] (axis cs:-86.1535,39.7645) to (axis cs:-86.1495,39.7645);
\draw[line width=0.3mm, subregioncolor1,-,>=latex] (axis cs:-86.1495,39.7645) to (axis cs:-86.1495,39.763);
\draw[line width=0.3mm, subregioncolor1,-,>=latex] (axis cs:-86.1495,39.763) to (axis cs:-86.1485,39.763);
\draw[line width=0.3mm, subregioncolor1,-,>=latex] (axis cs:-86.1485,39.763) to (axis cs:-86.1485,39.769);
\draw[line width=0.3mm, subregioncolor1,-,>=latex] (axis cs:-86.1485,39.769) to (axis cs:-86.149,39.769);
\draw[line width=0.3mm, subregioncolor1,-,>=latex] (axis cs:-86.149,39.769) to (axis cs:-86.149,39.7705);
\draw[line width=0.3mm, subregioncolor1,-,>=latex] (axis cs:-86.149,39.7705) to (axis cs:-86.151,39.7705);
\draw[line width=0.3mm, subregioncolor1,-,>=latex] (axis cs:-86.151,39.7705) to (axis cs:-86.151,39.7725);
\draw[line width=0.3mm, subregioncolor1,-,>=latex] (axis cs:-86.151,39.7725) to (axis cs:-86.1495,39.7725) ;
\draw[line width=0.3mm, subregioncolor1,-,>=latex] (axis cs:-86.1495,39.7725) to (axis cs:-86.1495,39.777);
\draw[line width=0.3mm, subregioncolor1,-,>=latex] (axis cs:-86.1495,39.777) to (axis cs:-86.151,39.777);
\draw[line width=0.3mm, subregioncolor1,-,>=latex] (axis cs:-86.151,39.777) to (axis cs:-86.151,39.776);
\draw[line width=0.3mm, subregioncolor1,-,>=latex] (axis cs:-86.151,39.776) to (axis cs:-86.155,39.776);
\draw[line width=0.3mm, subregioncolor1,-,>=latex] (axis cs:-86.155,39.776) to (axis cs:-86.155,39.777);
\draw[line width=0.3mm, subregioncolor1,-,>=latex] (axis cs:-86.155,39.777) to (axis cs:-86.156,39.777);
\draw[line width=0.3mm, subregioncolor1,-,>=latex] (axis cs:-86.156,39.777) to (axis cs:-86.156,39.776);
\draw[line width=0.3mm, subregioncolor1,-,>=latex] (axis cs:-86.156,39.776) to (axis cs:-86.1605,39.776);
\draw[line width=0.3mm, subregioncolor1,-,>=latex] (axis cs:-86.1605,39.776) to (axis cs:-86.1605,39.773);
%\draw[line width=0.3mm, subregioncolor1,-,>=latex] (axis cs:-86.159,39.776) to (axis cs:-86.1545,39.776);
%\draw[line width=0.3mm, subregioncolor1,-,>=latex] (axis cs:-86.157,39.775) to (axis cs:-86.157,39.7725);
%\draw[line width=0.3mm, subregioncolor1,-,>=latex] (axis cs:-86.157,39.7725) to (axis cs:-86.166,39.7725);
\draw[line width=0.3mm, subregioncolor1,-,>=latex] (axis cs:-86.1605,39.773) to (axis cs:-86.166,39.773);
\draw[line width=0.3mm, subregioncolor1,-,>=latex] (axis cs:-86.166,39.773) to (axis cs:-86.166,39.7744);
\draw[line width=0.3mm, subregioncolor1,-,>=latex] (axis cs:-86.166,39.7744) to (axis cs:-86.1668,39.7744);
\fi
				%\node[text=subregioncolor1] at (axis cs:-86.164,39.775) {R1A};
				%}
			
		\end{axis}
%	\pgfresetboundingbox
%	\path
%	(current axis.south west) -- ++(-0in,-0in)
%	rectangle (current axis.north east) -- ++(0in,0in);
	\end{tikzpicture}
	
\end{document}